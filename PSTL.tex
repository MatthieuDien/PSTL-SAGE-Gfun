\documentclass[12pt]{article}
\usepackage[utf8]{inputenc}
\usepackage[french]{babel}
\usepackage[T1]{fontenc}                      

\usepackage{array}
\usepackage{rotating}
\usepackage{fullpage}
\usepackage{graphicx}
\usepackage{hyperref}
\usepackage{dsfont}
\usepackage[french]{algorithm2e}

\usepackage{amsmath}
\usepackage{amssymb}
\usepackage{txfonts}
\usepackage[amsmath,thmmarks,thref,hyperref]{ntheorem}
\usepackage[all]{xy}
\frenchspacing
\usepackage{bbold}
\usepackage{pifont}

\hypersetup{
      backref=true,    
     pagebackref=true,
     hyperindex=true,
     colorlinks=true,
     breaklinks=true,
     urlcolor= blue,
     linkcolor= black,
     citecolor=black,
     bookmarks=true,
     bookmarksopen=true,
     pdftitle={rapport},
     pdfauthor={Marguerite Zamansky},
     }
     
\newtheorem*{definition}{Définition}

\title{Outils de combinatoire analytique en Sage}
\author{Matthieu Dien, Marguerite Zamansky\\
encadrés par Antoine Genitrini et Frederic Peschanski }    
\begin{document}
     
     \maketitle
     
\section{Introduction}
Là, je raconte le projet, j'explique en gros ce qu'on fait.
\subsection{Quelques bases de combinatoire analytique}
La combinatoire analytique a pour but de décrire de manière quantitatives des structures combinatoires en utilisant des outils analytiques. Il faut que j'ajoute des trucs ici.
\begin{definition}
  Une \underline{classe combinatoire} est un ensemble fini ou dénombrable sur lequel est défini une fonction taille qui vérifie les conditions suivantes : 
  \begin{itemize}
    \item la taille d'un élément est un entier positif ,
    \item il y a un nombre fini d'élément de chaque taille.
  \end{itemize}
\end{definition}

\begin{definition}
La \underline{suite de comptage} d'une classe combinatoire $\mathcal A$ est la suite $(A_n)_{n \geqslant 0}$ où $A_n$ est le nombre d'objet de taille $n$ dans $\mathcal A$.
\end{definition}

\begin{definition}
  La \underline{série génératrice ordinaire} d'une suite $(A_n)$ est la série entière
  $$A(z) = \sum_{n=0}^{\infty} A_n z^n .$$
  La série génératrice ordinaire d'une classe combinatoire est la série génératrice ordinaire de sa suite de comptage. De manière équivalente, le série génératrice d'une classe combinatoire $\mathcal A$ peut s'écrire sous la forme 
  $$A(z) = \sum_{a \in \mathcal A} z ^{|a|} .$$
\end{definition}

Connaître cette série, c'est donc connaître le nombre d'objet de chaque taille de la classe combinatoire étudiée. Pour connaître le nombre d'objet de taille $n$, il faut extraire le $n$-ième coefficient de cette série génératrice. Si on note $f(z)$ la série entière$\sum f_n z^n$, on note $[z^n]f(z)$ l'extraction du coefficient de $z^n$.

exemple partition d'entier
\subsection{Fonctions génératrices multivariées}

\section{Sage}

\subsection{Représentation des séries}
Lazy, Stream et tout ça, représentation…
\subsection{Opérateurs}
\subsubsection{Addition}
\subsubsection{Multiplication}
\subsubsection{Séquence}
\end{document}