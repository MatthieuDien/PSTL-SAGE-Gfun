\documentclass{beamer}
\mode<presentation>
{\usetheme[left]{Goettingen}
\useinnertheme[shadow]{rounded}
}
\usefonttheme{serif}

\usepackage[french]{babel}

\usepackage[applemac]{inputenc}

\usepackage{times}
\usepackage[T1]{fontenc}
\usepackage{array}
\usepackage{rotating}
\usepackage{graphicx}
\usepackage{hyperref} 


\usepackage{amsmath}
\usepackage{amssymb}
\usepackage{txfonts}

\usepackage{color}
\usepackage{multicol}
\usepackage{multirow}

\hypersetup{
     backref=true,    %permet d'ajouter des liens dans...
     pagebackref=true,%...les bibliographies
     hyperindex=true, %ajoute des liens dans les index.
     colorlinks=true, %colorise les liens
     breaklinks=true, %permet le retour à la ligne dans les liens trop longs
     urlcolor= blue,  %couleur des hyperliens
     linkcolor= , %couleur des liens internes
     bookmarks=true,  %créé des signets pour Acrobat
     bookmarksopen=true,            %si les signets Acrobat sont créés,
                                    %les afficher complètement.
     pdftitle={Pr�sentation colour numbers of complete graphs}, %informations apparaissant dans
     pdfauthor={Marguerite Zamansky},     %dans les informations du document
     }



\title	
{Outils de combinatoire analytique en sage}
\subtitle{Projet STL}

\author
{Matthieu Dien \and Marguerite Zamansky}

\institute
{Universit� Pierre et Marie Curie}

\date{\today}

\begin{document}

\begin{frame}
\titlepage
\end{frame}

\begin{frame}{S�ries g�n�ratrices multivari�es}

On parle des s�ries multivari�es

Exemples : arbres binaires-ternaires, profondeur cumul�e sur arbre binaire.
Avec les sp�cifications.
\end{frame}

\begin{frame}{Sage}
\begin{itemize}
\item logiciel libre de calcul formel et num�rique
\pause
\item avec une interface en python.
\end{itemize}
\end{frame}

\begin{frame}{Impl�mentation}
\begin{block}{Formal multivariate power series}
\begin{itemize}
\item Bas� sur le travail fait sur les s�ries g�n�ratrices monovari�es.
\pause
\item Repr�sentation m�moire sous forme de stream.
\end{itemize}
\end{block}
%\includegraphics[]{unpetitdessindestream.pdf}
\end{frame}

\begin{frame}{D�monstration}
\begin{block}{sage block}
La sp�cification
calcul des coefficients
\end{block}
\end{frame}

\begin{frame}{Conclusion}
\begin{itemize}
\item Patch bug dans Sage
\pause
\item Proposition du package
\pause
\item Continuer l'impl�mentation pour avoir les fonctionnalit�s disponibles dans Gfun
\end{itemize}
\end{frame}

\begin{frame}{}
Merci � Antoine Genitrini et Fr�d�ric Peschanski
\bigskip

\textit{\url {www.sagemath.org}}
\bigskip

Analytic Combinatorics, Philippe Flajolet et Robert Sedgewick
\end{frame}
\end{document}
