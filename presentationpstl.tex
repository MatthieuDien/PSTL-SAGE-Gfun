\documentclass{beamer}
\mode<presentation>
{\usetheme[left]{Goettingen}
\useinnertheme[shadow]{rounded}
}
\usefonttheme{serif}

\usepackage[french]{babel}

\usepackage[applemac]{inputenc}

\usepackage{times}
\usepackage[T1]{fontenc}
\usepackage{array}
\usepackage{rotating}
\usepackage{graphicx}
\usepackage{hyperref} 


\usepackage{amsmath}
\usepackage{amssymb}
\usepackage{txfonts}

\usepackage{color}
\usepackage{multicol}
\usepackage{multirow}

\hypersetup{
     backref=true,    %permet d'ajouter des liens dans...
     pagebackref=true,%...les bibliographies
     hyperindex=true, %ajoute des liens dans les index.
     colorlinks=true, %colorise les liens
     breaklinks=true, %permet le retour à la ligne dans les liens trop longs
     urlcolor= blue,  %couleur des hyperliens
     linkcolor= , %couleur des liens internes
     bookmarks=true,  %créé des signets pour Acrobat
     bookmarksopen=true,            %si les signets Acrobat sont créés,
                                    %les afficher complètement.
     pdftitle={Pr�sentation colour numbers of complete graphs}, %informations apparaissant dans
     pdfauthor={Marguerite Zamansky},     %dans les informations du document
     }



\title	
{Outils de combinatoire analytique en sage}
\subtitle{Projet STL}

\author
{Matthieu Dien \and Marguerite Zamansky}

\institute
{Universit� Pierre et Marie Curie}

\date{\today}

\begin{document}

\begin{frame}
\titlepage
\end{frame}

\begin{frame}{D�finitions}

\begin{block}{Classe Combinatoire}
Une classe combinatoire $\mathcal{A}$ est un ensemble muni d'une application taille $ | \cdot | : \mathcal{A} -> \mathbb{N} $
tel que $$ \forall n \in \mathbb{N} , \{ a \in \mathcal{A} , |a| = n \} \text{ est fini} $$
\end{block}

\begin{block}{Param�tre Scalaire}
Un param�tre scalaire $\chi$  de $\mathcal{A}$  est une fonction surjective de $\mathcal{A}$ dans $\mathbb{N}$
\end{block}

\end{frame}

\begin{frame}{D�finitions}{(suite)}

\begin{block}{S�rie g�n�ratrice multivari�e}
Une s�rie g�n�ratrice $A$ associ�e � une classe combinatoire $\mathcal{A}$ et $k$ param�tres scalaires $\chi_j$ : 
$$ A(X_1 \ldots X_k) = \sum_{i_1, \ldots ,i_k \ge 0}^{+\infty} a_{i_1, \ldots , i_k} \ X_1^{i_1} \ldots X_k^{i_k} $$
permet de compter le nombre d'�l�ment de $\mathcal{A}$ : $a_{i_1, \ldots , i_k} = \mathrm{Card}( e \in \mathcal{A}, \  \chi_j(e)=i_j, \ \forall j \in \llbracket 1 , k \rrbracket )$
\end{block}

\end{frame}


\begin{frame}{S\'eries G\'en\'eratrices Multivari\'ees}{Exemple (1)}

Arbres binaire-ternaire : \newline
$$ ABT(z,u,v,w) = z \cdot w \ + \ u \cdot w \cdot ABT^2(z,u,v,w) \ + \  v \cdot w \cdot ABT^3(z,u,v,w) $$
\newline
\begin{description}
\item[z] : les feuilles
\item[u] : les n\oe{}uds binaires
\item[v] : les n\oe{}uds ternaires
\item[w] : la taille totale
\end{description}

\end{frame}

\begin{frame}{S\'eries G\'en\'eratrices Multivari\'ees}{Exemple (2)}

\begin{description}
\item[Arbre de Catalan] : un arbre g�n�ral (enracin� et planaire)
\item[Longueur de cheminement] : somme des distances entre chaque noeud et la racine d'un arbre
\end{description}

\pause

\bigskip

Longueur de cheminement d'un arbre de Catalan :\newline
$$ AG(z,u) = z \cdot SEQ(AG(zu,u)) $$
\begin{description}
\item[z] : le nombre de n\oe{}uds
\item[u] : longueur de cheminement
\end{description}

%Exemples : arbres binaires-ternaires, profondeur cumul\'ee sur arbre binaire.
%Avec les sp\'ecifications.

\end{frame}


\begin{frame}{Sage}

\begin{figure}
\includegraphics[width=2cm]{sage_logo_new.png}
\end{figure}

\begin{itemize}
\item logiciel libre de calcul formel et num\'erique
\pause
\item regroupe des outils d�j� connus et �prouv�s (GP/PARI, GAP, Singular, Maxima)
\pause
\item et ses propres paquets (combinat, rings, matrix ...)
\pause
\item le tout interfac� par un top-level Python
\end{itemize}

\end{frame}




\begin{frame}{Impl\'ementation}
\begin{block}{Formal multivariate power series}
\begin{itemize}
\item Bas\'e sur le travail fait sur les s\'eries g\'en\'eratrices monovari\'ees.
\pause
\item Repr\'esentation m\'emoire sous forme de stream.
\end{itemize}
\end{block}
%\includegraphics[]{unpetitdessindestream.pdf}
\end{frame}

\begin{frame}{D�monstration}
\begin{block}{sage block}
La sp\'ecification
calcul des coefficients
\end{block}
\end{frame}

\begin{frame}{Conclusion}
\begin{itemize}
\item Patch bug dans Sage
\pause
\item Proposition du package
\pause
\item Continuer l'impl\'ementation pour avoir les fonctionnalit\'es disponibles dans Gfun
\end{itemize}
\end{frame}

\begin{frame}{}
Merci \`a Antoine Genitrini et Fr\'ed\'eric Peschanski
\bigskip

\textit{\url {www.sagemath.org}}
\bigskip

Analytic Combinatorics, Philippe Flajolet et Robert Sedgewick
\end{frame}
\end{document}
